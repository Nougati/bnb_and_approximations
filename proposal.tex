\documentclass[12pt, a4paper]{article}
\usepackage[utf8]{inputenc}
\usepackage{graphicx}
\graphicspath{{images/}}

\begin{document}
\begin{titlepage}
  \centering
  {\scshape\LARGE Monash University \par}
  \vspace{1cm}
  {\scshape\Large Project Proposal \par}
  \vspace{1.5cm}
  {\huge Implicit enumeration with dual bounds from approximation algorithms\par}
  \vspace{2cm}
  {\Large Nelson Frew\par}
  \vfill
  supervised by\par
  Dr. Pierre Le Bodic
  \vfill
  {\large \today\par}
\end{titlepage}
\tableofcontents
\newpage
\section{Introduction}
In Mixed Integer Programming (MIP), the state-of-the-art method for obtaining a solution is Branch and Bound. Branch and bound is a recursive divide-and-conquer approach which finds an optimal value by implicit enumeration of a search space. The search is guided by \emph{primal} and \emph{dual} bounds on the optimal value, which are obtained from solving a relaxed form of the original problem. One achieves implicit enumeration by constraining variables to particular values (known as \emph{branching} on variables) and obtaining bounds on the optimal value within this subspace. If the subspace's bounds indicate our best value so far cannot be improved, we remove it from our search (known as \emph{pruning}). If we find a feasible solution to our original problem which improves on the best value we have to that point, we move to another variable branch, using this new best solution to inform pruning of future search spaces. Traditionally, we relax a Mixed-Integer Program by removing the integrality constraint and solving the produced Linear Program (known as \emph{linear programming bounding}), however such bounding strategies can return arbitrarily poor results, which can seriously impact performance. However, there exists a method to specifically address this limitation: approximation algorithms (AAs).  

In this project, we will investigate methods of effectively using bounds provided by AAs, for problems where valid approximations exist. We aim to evolve algorithm designs and settings that will improve performance with respect to this strategy. This may include include branching strategies, node selection schemes, and incorporating warm starting into AAs. We aim to find ways to leverage the structure of this approach to cut down compute-time.

As mentioned, LP bounding has no guarantee on quality: a key limitation inherent to the strategy which our project addresses specifically. Further, with AAs, one can derive both primal and dual bounds simultaneous. Finally, should this strategy prove promising, it is also possible that a new direction for both optimisation and approximations researchers may be revealed. 

MIP solvers are used in almost every industry in real world optimisation problems, with applications in advertising, portfolio management, and numerous scheduling problems. It is then in the interest of researchers, developers, businesses that continued research is done to improve the efficiency of these solvers. By targeting a key bottleneck in MIP solver computation, this project is a concern of such stakeholders across all industries. 

\section{Research Context / Background}

For our purposes, we will define a \textit{heuristic} as an algorithm which provides a feasible solution to a problem with emphasis on speed. Heuristics achieve this by trading an optimality guarantee for a faster run time, which may be unrealistically long if solved to optimality, however these may be arbitrarily poor.

We distinguish an AA as a heuristic which has a worst case guarantee on the solution returned. By introducing an error parameter $\epsilon$, it is sometimes possible to parameterise the guarantee of the approximation while having a variable impact on the compute time. In such a case, for a maximisation problem, one example of bounds on the quality of the approximate solution $z_h$, relative to optimal solution $z$, is $(1-\epsilon)\cdot z < z_h < z$.   

The context for this project's direction follows from the chronology of MIP and AA research alike. Solving MIP problems was revolutionised after the Branch and Bound algorithm was first proposed in 1960 \cite{LandDoig}, which became a seminal paper in the field. As research progressed further with this new scheme, a new field of AAs was opening up during the 1970's with works on Knapsack \cite{IbarraKim}, TSP \cite{CristofidesTSP}, Facility-location \cite{CornuejolsFisherNemhauser}, among many others, all appearing in quick succession. Following this, in 1980, Wolsey \cite{WOLSEY} proposed a method of combining the findings from approximation algorithms with branch and bound: one of the few treatments of the topic that exists today. In his paper, Wolsey posits a general AA worst-case analysis technique, and applies what is found to create implicit enumeration for a branch and bound with both AA bounds and LP relaxations. 

The basis for this project follows from specifically this point, and acknowledges one key property of AAs: by using the AA bounds, we can not only obtain the \textit{a priori} (inferred) dual bounds, we can use analysis techniques to also derive \textit{a posteriori} primal bounds with guarantees. In doing this, we omit the need for the use of LPs in successfully carrying out a branch and bound search.

For the beginning of this project, we explore concepts of approximations within branch and bound, utilising the Knapsack problem as a case study. To do this, a \textit{Fully Polynomial Time Approximation Scheme} (FPTAS) is introduced for Knapsack in \cite{BOOK:2}, where the reader is referred to for formal definitions and proofs. A key point is that we are interested in exploring FPTAS's which use dynamic programming (DP). Since DP algorithms lend themselves to partially computed problems, we are interested in also warm starting subproblems with these. By definition, FPTAS's adhere to our definition of an AA.

\section{Research Design}

\subsection{Methodologies}

The aim of this research is to iteratively explore methods of improving the performance of our branch and bound algorithm. Specifically we are asking \textbf{can we use approximation algorithms to compute dual bounds effectively in a branch and bound scheme for combinatorial optimisation problems?}\\

Our research methodology will be driven by systematically addressing the following questions:

\begin{itemize}
\item What branching, searching, and pruning strategies does this strategy benefit most from?
\item Given that we are looking at FPTAS's with DP algorithms, to what extent does this approach lend itself to the concept of warm starting in branch and bound?
\item What relationships, if any, exist between problem types and this method?
\end{itemize}

As the project progresses, these questions will develop into subquestions themselves as further information is understood.

We will use automated performance logging and characterise the approaches based on this. Comparisons will be done incrementally so to emphasise the effects of any changes made to the algorithm.

Particularly, the research methodology will follow a \textit{design science} philosophy. This entails a series of systematic re-evaluations of artifacts in light of research questions, which will take place throughout the duration of the project. 

\subsection{Proposed Thesis chapter headings}
We expect the thesis to follow this general format:
\begin{enumerate}
  \item Abstract
  \item Introduction
  \item Literature review
  \item Method section: Initial parameterisation and results
  \item Method section: Algorithm revision and results 
  \item Method section: Fully improved approach
  \item Discussion
  \item Conclusion
  \item Bibliography / References
\end{enumerate}
\newpage
\subsection{Timetable}
The proposed timetable for successful completion of this project is as follows:

\begin{center}
  \begin{tabular}{||c | c||}
  \hline
  Date & Milestone \\ [0.5ex]
  \hline\hline
  16/04 & Explore branching \\
  \hline
  28/05 & Explore node selection \\
  \hline
  04/06 & Interim presentation, Lit. Rev  \\
  \hline
  30/07 & Explore warm starting \\
  \hline
  27/08 & Generalisations of method \\
  \hline
  10/09 & Problem type evaluation \\
  \hline
  01/10 & Thesis drafting process \\
  \hline
  29/10 & Final presentation \\
  \hline 
  05/11 & Thesis submission \\ [1ex]
  \hline
  \end{tabular}
\end{center}

Of course, as the project progresses, this timetable may be subject to change.

\subsection{Potential Difficulties}
As is the case with any research project, there are some setbacks which should be acknowledged.
\begin{itemize}
  \item Insufficient planning
  \item Ineffective idea development
  \item Substandard adoption of secondary texts
  \item Poor practice in artifact development
  \item Improper styling techniques
\end{itemize}

\section{Deliverables / Outcomes}
We expect there to be a series of deliverables associated with the completion of this project. In addition to this proposal, a literature review, an interim and final presentation, and a final thesis will be provided. Moreover, as is expected with design science research approaches, we aim to have publicly available artifacts for each point in the research, hosted online. It is hoped that the mechanisms will be available as a library that can be freely used.

\section{Bibliography / References}
\bibliography{proposal}
\bibliographystyle{plain}
\end{document}
