\documentclass[12pt, a4paper]{article}
\usepackage[utf8]{inputenc}

\begin{document}
\begin{titlepage}
  \centering
  {\scshape\LARGE Monash University \par}
  \vspace{1cm}
  {\scshape\Large Project Proposal \par}
  \vspace{1.5cm}
  {\huge Using approximation algorithms with Branch and Bound optimisation problems\par}
  \vspace{2cm}
  {\Large Nelson Frew\par}
  \vfill
  supervised by\par
  Dr. Pierre Le Bodic
  \vfill
  {\large \today\par}
\end{titlepage}
\tableofcontents
\newpage
\section{Introduction}
We aim to investigate the use of approximation algorithms within a branch and bound scheme. In particular, we are focused on using approximations as a method of providing dual bounds which have a worst case bounding. 

This project is concerned with using such methods to explore new paths in general optimisation problem solving. Since such an approach is unique, we seek to explore configurations of this branch and bound with approximation which achieve efficient running times. The major points of interest for doing this include optimal branching strategies, node selection schemes, and incorporating warm starting.

Our objective is to craft an approach, using this scheme, to solve branch and bound optimisation problems quickly. If this is achieved, a new promising direction of research for optimisation researchers and approximations researchers may be revealed. Moreover, this project investigates a method with a worst-case guarantee on dual bounds, a key problem in traditional LP relaxations.

MIP solvers are used in almost every industry in real world optimisation problems, due to the ubiquity of a need for both resource minimisation (expenditure) and maximisation (turnover). A high quality MIP solver is, then, a concern of businesses and developers alike. In investigating this new avenue for research, we have the potential to directly influence stakeholders in industries worldwide (THIS IS SO LOFTY).

\section{Research Context / Background}

Define B\&B from Conforti

Define AA + Heuristic, visiting Vasirani along the way.

\section{Research Design}

\subsection{Methodologies}

Sort of a scientific approach sort of thing

\subsection{Proposed Thesis chapter headings}
Abstract, introduction, literature review, the pledge, the turn, the prestige, discussion, conclusion, counselling for people recovering from our sweet showmanship
\subsection{Timetable}
13th of April - Proposal
25th of May - Literature Review / Interim Presentation
<Whatever day S2, Wk14 is> Final Presentation
<Whatever day S2, Wk15 is> Thesis

\subsection{Potential Difficulties}
\begin{itemize}
  \item Weak organisation (Reword this because it's copy-pasted)
  \item Poor development of ideas (Reword this because it's copy-pasted)
  \item Weak use of secondary sources (Reword this because it's copy-pasted)
  \item Excessive errors (Reword this because it's copy-pasted)
  \item Stylistic Weaknesses (Reword this because it's copy-pasted)
\end{itemize}

\section{Deliverables / Outcomes}
How do I use bibtek?~\cite{BOOK:1}
\section{Bibliography / References}
\bibliography{proposal}
\bibliographystyle{plain}
\end{document}
