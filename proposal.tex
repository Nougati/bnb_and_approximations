\documentclass[12pt, a4paper]{article}
\usepackage[utf8]{inputenc}
\usepackage{graphicx}
\graphicspath{{images/}}

\begin{document}
\begin{titlepage}
  \centering
  {\scshape\LARGE Monash University \par}
  \vspace{1cm}
  {\scshape\Large Project Proposal \par}
  \vspace{1.5cm}
  {\huge Implicit enumeration with dual bounds from approximation algorithms\par}
  \vspace{2cm}
  {\Large Nelson Frew\par}
  \vfill
  supervised by\par
  Dr. Pierre Le Bodic
  \vfill
  {\large \today\par}
\end{titlepage}
\tableofcontents
\newpage
\section{Introduction}
In Mixed Integer Programming (MIP), the state-of-the-art method for obtaining a solution is Branch and Bound. Branch and bound is a recursive divide-and-conquer approach which finds an optimal value by implicit enumeration of a search space. The search is guided by \emph{primal} and \emph{dual} bounds on the optimal value, which are obtained from solving a relaxed form of the original problem. One achieves implicit enumeration by constraining variables to particular values (known as \emph{branching} on variables) and obtaining bounds on the optimal value within this subspace. If the subspace's bounds indicate our best value so far cannot be improved, we remove it from our search (known as \emph{pruning}). If we find a feasible solution to our original problem which improves on the best value we have to that point, we move to another variable branch, using this new best solution to inform pruning of future search spaces. Traditionally, we relax a Mixed-Integer Program by removing the integrality constraint and solving the produced Linear Program (known as \emph{linear programming bounding}), however such bounding strategies can return arbitrarily poor results, which can seriously impact performance. However, there exists a method to specifically address this limitation: approximation algorithms (AAs).  

In this project, we will investigate methods of effectively using bounds provided by AAs, for problems where valid approximations exist. We aim to evolve algorithm designs and settings that will improve performance with respect to this strategy. This may include include branching strategies, node selection schemes, and incorporating warm starting into AAs. We aim to find ways to leverage the structure of this approach to cut down compute-time.

As mentioned, LP bounding has no guarantee on quality: a key limitation inherent to the strategy which our project addresses specifically. Further, with AAs, one can derive both primal and dual bounds simultaneous. Finally, should this strategy prove promising, it is also possible that a new direction for both optimisation and approximations researchers may be revealed. 

MIP solvers are used in almost every industry in real world optimisation problems, with applications in advertising, portfolio management, and numerous scheduling problems. It is then in the interest of researchers, developers, businesses that continued research is done to improve the efficiency of these solvers. By targeting a key bottleneck in MIP solver computation, this project is a concern of such stakeholders across all industries. 

\section{Research Context / Background}

For our purposes, we will define a \textit{heuristic} as an algorithm which provides a feasible solution to a problem with emphasis on speed. Heuristics can quickly return a solution to a problem which may take too long when solved to optimality. A key caveat, however, is that heuristic solutions can be arbitrarily poor. AAs extend heuristics by providing a worst case guarantee on the solution returned. If an AA returns a solution $z$ guaranteed to be within a constant factor $\alpha$ of the optimal value $OPT$, such that $z \in [\alpha \cdot OPT, OPT]$ (for a maximisation problem) we call it a \emph{constant ratio approximation scheme}. If such a scheme runs in time proportional to a polynomial in the instance size, and $\alpha = (1-\epsilon)$ for error parameter $\epsilon > 0$, the approximation is a \emph{polynomial time approximation scheme} (PTAS), Further, if a PTAS runs in time proportional to both the instance size and $\frac{1}{\epsilon}$, it is known as a \emph{fully polynomial time approximation scheme} (FPTAS). For a detailed description of such approximation schemes and their associated proofs, the reader is directed to \cite{BOOK:2} for further reading.


Following Dantzig's \cite{Dantzig} Simplex method contribution in 1947, the problem of solving efficiently LPs to optimality had been sufficiently addressed within the fields of Engineering and Applied Mathematics. However, solving discrete optimisation problems, known as Integer Programs (IPs), remained an open problem. Research splintered following this: exactly solving IPs and active pusuit of polynomial time algorithms for hard problems and their variants. Research into optimally solving IPs led to Land and Doig's initial branch and bound method \cite{LandDoig} in 1960, but had little application due to limitations in computing hardware at the time. On the other hand, in a response to the growing category of NP-hard problems which IP was a part of, Computational Complexity Theory research began. Concerns with finding polynomially bounded algorithms led to compromising optimality for efficiency, in what became AA research through the 1970s. Approximations for Knapsack \cite{IbarraKim}, TSP \cite{CristofidesTSP}, Facility-location \cite{CornuejolsFisherNemhauser}, and many others were devised through the decade. Responding to the progress of both fields, in 1980 Wolsey \cite{WOLSEY} produced a general analysis technique attempting to unify AAs of the 1970s with the Branch and Bound algorithm of 1960. Wolsey provided a general analysis technique related approximation worst-cases with optimal LP relaxation solutions, and devised a branch and bound procedure from these components. However, since this point, to our knowledge, no further work has been done in investigating the relationship between approximations and branch and bound. It is not known whether a stronger link exists.

\section{Research Design}

\subsection{Methodologies}

The aim of this research is to iteratively explore methods of improving the performance of our branch and bound algorithm. Specifically we are asking \textbf{can we use approximation algorithms to compute dual bounds effectively in a branch and bound scheme for combinatorial optimisation problems?}\\

Our research methodology will be driven by systematically addressing the following questions:

\begin{itemize}
\item What branching, searching, and pruning strategies does this strategy benefit most from?
\item For various problems and their respective AAs, how well can this strategy perform?
\item What relationships, if any, exist between problem types and this method?
\end{itemize}

As the project progresses, these questions will develop into subquestions themselves as further information is obtained.

We will use automated performance logging and characterise the approaches based on this. Comparisons will be done incrementally so to emphasise the effects of any changes made to the algorithm.

Particularly, the research methodology will follow a \textit{design science} philosophy. This entails a series of systematic re-evaluations of artifacts in light of research questions, which will take place throughout the duration of the project. 

\subsection{Proposed Thesis chapter headings}
We expect the thesis to follow this general format:
\begin{enumerate}
  \item Abstract
  \item Introduction
  \item Literature review
  \item Method section: Initial parameterisation and results
  \item Method section: Algorithm revision and results 
  \item Method section: Fully improved approach
  \item Discussion
  \item Conclusion
  \item Bibliography / References
\end{enumerate}
\subsection{Timetable}
The proposed timetable for successful completion of this project is as follows:

\begin{center}
  \begin{tabular}{||c | c||}
  \hline
  Date & Milestone \\ [0.5ex]
  \hline\hline
  16/04 & Explore branching \\
  \hline
  28/05 & Explore node selection \\
  \hline
  04/06 & Interim presentation, Lit. Rev  \\
  \hline
  30/07 & Explore warm starting \\
  \hline
  27/08 & Generalisations of method \\
  \hline
  10/09 & Problem type evaluation \\
  \hline
  01/10 & Thesis drafting process \\
  \hline
  29/10 & Final presentation \\
  \hline 
  05/11 & Thesis submission \\ [1ex]
  \hline
  \end{tabular}
\end{center}

Of course, as the project progresses, this timetable may be subject to change.

\subsection{Potential Difficulties}
As is the case with any research project, there are potential setbacks which should be acknowledged. Further, work on the project until this point has indicated that the following issues have the capacity to disrupt research:
\begin{itemize}
  \item Adaptations of known AAs to our scheme; while some methods are known, in optimising and weaving individual AAs may be a key issue.
  \item The bridging of theoretically proven approximation algorithms to practical applications: algorithms specified for theoretical conclusions can predicate themselves on design choices unsuitable for practical application.
  \item Development of key project questions and ideas not addressed in adequate depth. This could be a result of both poor understanding and issues with planning.
  \item Application and artifact development which is predicated on incorrect theoretical understanding, i.e. implementing something without full ability to give justification.
  \item Substandard adoption of secondary texts related to specific functionality, leading to oversights in a particular artifact's design. This relates to not fully utilising research output of a particular problem's approximation schemes.
\end{itemize}

\section{Deliverables / Outcomes}
We expect there to be a series of deliverables associated with the completion of this project. In addition to this proposal, a literature review, an interim and final presentation, and a final thesis will be provided. Moreover, as is expected with design science research approaches, we aim to have publicly available artifacts for each point in the research, hosted online. It is hoped that the mechanisms will be available as a library that can be freely used.

\section{Bibliography / References}
\bibliography{proposal}
\bibliographystyle{plain}
\end{document}
