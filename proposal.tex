\documentclass[12pt, a4paper]{article}
\usepackage[utf8]{inputenc}

\begin{document}
\begin{titlepage}
  \centering
  {\scshape\LARGE Monash University \par}
  \vspace{1cm}
  {\scshape\Large Project Proposal \par}
  \vspace{1.5cm}
  {\huge Implicit enumeration with dual bounds from approximation algorithms\par}
  \vspace{2cm}
  {\Large Nelson Frew\par}
  \vfill
  supervised by\par
  Dr. Pierre Le Bodic
  \vfill
  {\large \today\par}
\end{titlepage}
\tableofcontents
\newpage
\section{Introduction}
We aim to investigate the use of approximation algorithms (AAs) within a branch and bound scheme. Specifically, we are focused on using approximations as a method of providing dual bounds with guarantees. 

To explore the merit of this approach, we will investigate possible configurations which improve performance. The major points of interest for this include optimal branching strategies, node selection schemes, and incorporating warm starting.

Our objective is to craft an approach to solve branch and bound optimisation problems quickly with dual bounds obtained from AAs. In achieving this, a new direction for both optimisation and approximations researchers may be revealed. Moreover, dual bounds from traditional LP relaxations can be arbitrarily poor, a key issue that this project addresses.

MIP solvers are used in almost every industry in real world optimisation problems, due to the ubiquity of a need for both resource minimisation (expenditure) and maximisation (turnover). It is in the interest of developers and businesses that continued research is done to improve the efficiency of these solvers. By targeting a key bottleneck in MIP solver computation, this project is a concern of such stakeholders across all industries. 

\section{Research Context / Background}

The branch and bound strategy is a well studied algorithm framework for combinatorial optimisation, and the reader is advised to use the description provided by \cite{BOOK:1} for a detailed description of the method.

For our purposes, we will define a \textit{heuristic} as an algorithm which provides a feasible, but not necessarily optimal, solution to a problem. Heuristics allow us to trade an optimality guarantee for a faster run time, which may be unrealistically long if solved to optimality.

We will now define an \textit{AA} as a heuristic which has a worst case guarantee on the solution returned. By introducing an error parameter $\epsilon$, it is sometimes possible to improve the guarantee of the approximation by accepting performance penalties. In such a case, for a maximisation problem, one example of bounds on the quality of the approximate solution $z_h$, relative to optimal solution $z$, is $(1-\epsilon)\cdot z < z_h < z$.   

Guarantees on heuristic qualities in the field of optimisation was specifically addressed in \cite{WOSLEY}, where a general heuristic analysis method is given. Worst case analyses of various heuristics for \textit{bin packing}, \textit{longest undirected hamilitonian tours}, among others, showed guarantees are heavily problem specific. Methods of embedding heuristics into branch and bound schemes are also proposed, showing that heuristics can work in tandem with LP relaxations in the construction of a branch and bound algorithm.

For this discussion, we will explore concepts of approximations within branch and bound, utilising the Knapsack problem as a case study. To do this, a \textit{Fully Polynomial Time Approximation Scheme} (FPTAS) is introduced for Knapsack in \cite{BOOK:2}, where the reader is referred to for formal definitions and proofs. By definition, FPTAS's adhere to our definition of an AA.



\section{Research Design}

\subsection{Methodologies}

The aim of this research is to iteratively explore methods of improving the performance of our branch and bound algorithm. Specifically we are asking \textbf{can we use approximation algorithms to compute dual bounds effectively in a branch and bound scheme for combinatorial optimisation problems?}

Our research methodology will be driven by systematically addressing the following questions:

\begin{itemize}
\item What branching, searching, and pruning strategies will improve our performance?
\item How can we exploit the concept of warm starting to evaluate bounds quicker?
\item Which problem types benefit most from the proposed approach?
\end{itemize}

As the project progresses, these questions will develop into subquestions themselves as further information is understood.

We will use automated performance logging and characterise the approaches based on this. Comparisons will be done incrementally so to emphasise the effects of any changes made to the algorithm.

\newpage
\subsection{Proposed Thesis chapter headings}
We expect the thesis to follow this general format:
\begin{enumerate}
  \item Abstract
  \item Introduction
  \item Literature review
  \item Method section: Initial parameterisation and results
  \item Method section: Algorithm revision and results 
  \item Method section: Fully improved approach
  \item Discussion
  \item Conclusion
  \item Bibliography / References
\end{enumerate}
\subsection{Timetable}
The proposed timetable for successful completion of this project is as follows:

\begin{center}
  \begin{tabular}{||c | c||}
  \hline
  Date & Milestone \\ [0.5ex]
  \hline\hline
  16/04 & Explore branching \\
  \hline
  28/05 & Explore node selection \\
  \hline
  04/06 & Interim presentation, Lit. Rev  \\
  \hline
  30/07 & Explore warm starting \\
  \hline
  27/08 & Generalisations of method \\
  \hline
  10/09 & Problem type evaluation \\
  \hline
  01/10 & Thesis drafting process \\
  \hline
  29/10 & Final presentation \\
  \hline 
  05/11 & Thesis submission \\ [1ex]
  \hline
  \end{tabular}
\end{center}

Of course, as the project progresses, this timetable may be subject to change.

\newpage
\subsection{Potential Difficulties}
As is the case with any research project, there are some setbacks which should be acknowledged.
\begin{itemize}
  \item Insufficient planning
  \item Ineffective idea development
  \item Substandard adoption of secondary texts
  \item Poor practice in artifact development
  \item Improper styling techniques
\end{itemize}

\section{Deliverables / Outcomes}
We expect there to be a series of deliverables associated with the completion of this project. In addition to this proposal, we will provide a literature review, an interim and final presentation, and a final thesis. Moreover, as is expected with design science research approaches, we aim to have publicly available artifacts for each point in the research, hosted online. It is hoped that the mechanisms will be available as a library that can be used.

\section{Bibliography / References}
\bibliography{proposal}
\bibliographystyle{plain}
\end{document}
