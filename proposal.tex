\documentclass[12pt, a4paper]{article}
\usepackage[utf8]{inputenc}

\begin{document}
\begin{titlepage}
  \centering
  {\scshape\LARGE Monash University \par}
  \vspace{1cm}
  {\scshape\Large Project Proposal \par}
  \vspace{1.5cm}
  {\huge Using approximation algorithms with Branch and Bound optimisation problems\par}
  \vspace{2cm}
  {\Large Nelson Frew\par}
  \vfill
  supervised by\par
  Dr. Pierre Le Bodic
  \vfill
  {\large \today\par}
\end{titlepage}
\tableofcontents
\newpage
\section{Introduction}
We aim to investigate the use of approximation algorithms within a branch and bound scheme. In particular, we are focused on using approximations as a method of providing dual bounds which have a worst case bounding. 

This project is concerned with using such methods to explore new paths in general optimisation problem solving. Since such an approach is unique, we seek to explore configurations of this branch and bound with approximation which achieve efficient running times. The major points of interest for doing this include optimal branching strategies, node selection schemes, and incorporating warm starting.

Our objective is to craft an approach, using this scheme, to solve branch and bound optimisation problems quickly. If this is achieved, a new promising direction of research for optimisation researchers and approximations researchers may be revealed. Moreover, this project investigates a method with a worst-case guarantee on dual bounds, a key problem in traditional LP relaxations.

MIP solvers are used in almost every industry in real world optimisation problems, due to the ubiquity of a need for both resource minimisation (expenditure) and maximisation (turnover). A high quality MIP solver is, then, a concern of businesses and developers alike. In investigating this new avenue for research, we have the potential to directly influence stakeholders in industries worldwide (THIS IS SO LOFTY).

\section{Research Context / Background}

The branch and bound strategy is a algorithm design framework for combinatorial optimisation, and has a wealth of research surrounding it. To define the strategy the reader is advised to use the description provided by \cite{BOOK:1}.

For our purposes, we will define a \textit{heuristic} as an algorithm which provides a feasible, but not necessarily optimal, solution to a problem. Heuristics allow us to trade accuracy for a faster run time, which may be undesirably large if solved to optimality.

We now define an \textit{approximation algorithm}, as a heuristic which has a worst case guarantee on the solution returned. That is, we can define strict bounds which we know the solution will reside. One can parameterise the tightness of these bounds with respect to an optimal solution, by selecting an error parameter we will refer to as $\epsilon$. In general, for a maximisation problem, the bounds on the quality of the approximate solution $z_h$, relative to optimal solution $z$, is $(1-\epsilon)\cdot z < z_h < z$.   

PUT WOSLEY HERE?!

For this discussion, we merge the concepts of approximations and branch and bound by addressing the Knapsack problem as a case study. A \textit{Fully Polynomial Time Approximation Scheme} (FPTAS) is introduced for Knapsack in \cite{BOOK:2}, where the reader is referred to for formal definitions and proofs. A proof is outlined, showing that the scheme adheres to our definition of an approximation algorithm.



\section{Research Design}

\subsection{Methodologies}

The aim of this research is to iteratively explore methods of improving the performance of our branch and bound algorithm. 

Our main question is: \textbf{can we use approximation algorithms to compute dual bounds effectively in a branch and bound scheme for combinatorial optimisation problems?}

In particular, our research methodology is driven by systematically addressing the following questions:

\begin{itemize}
\item What branching, searching, and pruning strategies will improve our performance?
\item How can we exploit the concept of warm starting to evaluate bounds quicker?
\item Which problem types benefit most from the proposed approach?
\end{itemize}

As the project progresses, these questions will develop into subquestions themselves as further information is understood.

We will use automated performance logging and characterise the approaches based on this. Comparisons will be done incrementally so to emphasise the effects of any changes made to the algorithm.

\subsection{Proposed Thesis chapter headings}
Abstract, introduction, literature review, the pledge, the turn, the prestige, discussion, conclusion, counselling for people recovering from our sweet showmanship
\subsection{Timetable}

13th of April - Proposal
25th of May - Literature Review / Interim Presentation
<Whatever day S2, Wk14 is> Final Presentation
<Whatever day S2, Wk15 is> Thesis

\subsection{Potential Difficulties}
\begin{itemize}
  \item Weak organisation (Reword this because it's copy-pasted)
  \item Poor development of ideas (Reword this because it's copy-pasted)
  \item Weak use of secondary sources (Reword this because it's copy-pasted)
  \item Excessive errors (Reword this because it's copy-pasted)
  \item Stylistic Weaknesses (Reword this because it's copy-pasted)
\end{itemize}

\section{Deliverables / Outcomes}
How do I use bibtek?~\cite{BOOK:1}
\section{Bibliography / References}
\bibliography{proposal}
\bibliographystyle{plain}
\end{document}
