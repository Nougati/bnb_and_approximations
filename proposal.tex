\documentclass[12pt, a4paper]{article}
\usepackage[utf8]{inputenc}
\usepackage{graphicx}
\graphicspath{{images/}}

\begin{document}
\begin{titlepage}
  \centering
  {\scshape\LARGE Monash University \par}
  \vspace{1cm}
  {\scshape\Large Project Proposal \par}
  \vspace{1.5cm}
  {\huge Implicit enumeration with dual bounds from approximation algorithms\par}
  \vspace{2cm}
  {\Large Nelson Frew\par}
  \vfill
  supervised by\par
  Dr. Pierre Le Bodic
  \vfill
  {\large \today\par}
\end{titlepage}
\tableofcontents
\newpage
\section{Introduction}
The field of discrete optimisation \cite{Hammer} focuses on solving problems that model discrete decision making. Mixed-Integer Programming (MIP) is a set of modelling and solving techniques for such discrete problems \cite{WolseyMIP}, where the objective to maximise/minimise $z = c x$, where $c$ contains problem data, and vector $x$ contains variables to be optimised. Constraints on optimising the objective function are generally linear, and variables can be specified to be integer or continuous. \cite{BOOK:1}

Within MIP, the state-of-the-art method for obtaining a solution is Branch-and-Bound \cite{Mitten}. Branch-and-Bound is a divide-and-conquer approach which finds an optimal value by what is known as \emph{implicit enumeration} of a search space. The search is guided by \emph{primal} and \emph{dual} bounds on the optimal value, which are obtained by finding potential to the original problem. For maximisation problems, the dual bound is an upper bound on the optimal solution, found by solving a relaxed form of the problem (a related, but computationally simpler version of the original problem). One achieves implicit enumeration by constraining variables to particular values (known as \emph{branching} on variables) and obtaining dual bounds on the optimal value within this subspace from the relaxation of the constrained problem. If the subspace's bounds indicate our best value for the original problem cannot be improved, we remove it from our search (this is known as \emph{bounding}). Alternatively, if while finding a subspace's dual bound, we find a better feasible solution to our original problem than we have seen, we update our best value to this and move to another variable branch, using this new best solution to inform pruning of future search spaces \cite{Gomory}. Traditionally, we relax a Mixed-Integer Program by removing the integrality constraints and solving the produced Linear Program. Such bounding strategies can return arbitrarily weak dual bounds, which can seriously impact performance. However, we propose to address this limitation can be addressed by providing dual bounds from another method: approximation algorithms (AAs).  

AAs give a guarantee on the value of their solution, if the problem is feasible. This means we obtain an \emph{a priori} primal bound on the solution, which use to inform our Branch-and-Bound. Crucially, however, we can analytically derive an \emph{a posteriori} dual bound on our problem, with information gained from the same AA.

In this project, we will investigate methods of effectively using bounds provided by AAs for problems where valid approximations exist. We aim to evolve algorithm designs and settings that will improve performance with respect to this strategy. To do this, we will investigate the use of MIP techniques which rely on LP, to see how they translate into use with AAs. This may include include branching strategies, node selection schemes, and incorporating warm starting into AAs \cite{Ralphs}. We aim to find ways to leverage the structure of this approach to cut down compute-time.

This project is of particular relevance to Branch-and-Bound research as a whole. This is in part due to the quality guarantee of dual bounds, and further because both primal and dual bounds may be found simultaneously. The project is of particular significance owing to the fact that should this strategy prove promising, a new direction for both optimisation and approximations researchers alike may be revealed. 

MIP solvers are used in almost every industry in real world optimisation problems, with applications in advertising, portfolio management, and numerous scheduling problems. It is then in the interest of researchers, developers, businesses that continued research is done to improve the efficiency of these solvers. By targeting a key bottleneck in MIP solver computation, this project is a concern of such key stakeholders across all industries. 

\section{Research Context / Background}

Following the inception of Dantzig's \cite{Dantzig} Simplex method in 1947, theoretical interest in finding optimal solutions shifted to harder problems. Solving discrete optimisation problems, known as Integer Programs (IPs), began to thrive as a focal point of research. Discrete optimisation gave rise to two new active research areas: exactly solving IPs, and active pusuit of polynomial time algorithms for such hard problems and their variants. Research into optimally solving IPs led to Land and Doig's initial Branch-and-Bound method \cite{LandDoig} in 1960, but had little direct applicability due to limitations in computing hardware at the time. On the other hand, as a response to the many discrete optimisation problems for which polynomial-time algorithms were not known, Computational Complexity Theory research began\cite{Cook}. Concerns with finding polynomially bounded algorithm run-times led to sacrificing optimality for efficiency, in what became AA research through the 1970s. Approximations for the Knapsack Problem\cite{IbarraKim}, Travelling Salesman Problem \cite{CristofidesTSP}, Facility Location problem \cite{CornuejolsFisherNemhauser}, and many others were devised through the decade. Until 1980, discrete problem algorithm design and AA design remained disconnected. It wasn't until Wolsey \cite{WOLSEY} attempting to unify AAs of the 1970s with the Branch-and-Bound algorithm of 1960 that the fields met again. Wolsey provided a general analysis technique relating approximation worst-cases with optimal LP relaxation solutions, and devised a Branch-and-Bound procedure from these components. However, since this point, to the best of our knowledge, no further work has been done in investigating the relationship between approximations and Branch-and-Bound. This project will investigate whether a stronger link exists and can be leveraged to improve performances. 

As mentioned, AAs are algorithms designed to quickly give a solution to a problem which is difficult when solved to optimailty. Importantly, they have a worst case guarantee on the solution returned. We will focus on \emph{constant-ratio AAs}, where if an AA returns a solution of value $z$ it is guaranteed to be within a constant factor $\alpha < 1$ of the optimal value $OPT$, such that $z \in [\alpha \cdot OPT, OPT]$ (for a maximisation problem). If such an algorithm runs in time proportional to a polynomial in the instance size, and $\alpha = (1-\epsilon)$ for error parameter $\epsilon > 0$, the approximation is a \emph{polynomial time approximation scheme} (PTAS). Further, if a PTAS runs in time proportional to both the instance size and $\frac{1}{\epsilon}$, it is known as a \emph{fully polynomial time approximation scheme} (FPTAS). For detailed descriptions, examples, and associated proofs of such approximation schemes, the reader is directed to \cite{BOOK:2} for further reading.

\section{Research Design}

\subsection{Methodologies}

The aim of this research is to iteratively explore methods of improving the performance of the Branch-and-Bound algorithm. Specifically we are asking \textbf{can we use approximation algorithms to compute dual bounds effectively in a Branch-and-Bound scheme for combinatorial optimisation problems?}\\
Our research methodology will be driven by systematically addressing the following questions:

\begin{itemize}
\item What branching, searching, and pruning strategies does this strategy benefit most from?
\item For various problems and their respective AAs, how well can this strategy perform?
\item What relationships, if any, exist between problem types and this method?
\end{itemize}

As the project progresses, these questions will develop into subquestions themselves as further information is obtained.

For the beginning of the project, however, we are first interested in providing a proof of concept on a simple problem. For this, we begin with the Knapsack \cite{PisingerKP} problem, and will branch into more nuanced problems with known approximations, as research progresses.

We will use automated performance logging and characterise the approaches based on this. Comparisons will be done incrementally so to emphasise the effects of any changes made to the algorithm.

Particularly, the research methodology will follow a \textit{design science} philosophy. This entails a series of systematic re-evaluations of artifacts in light of research questions, which will take place throughout the duration of the project. 

\subsection{Proposed Thesis chapter headings}
We expect the thesis to follow this general format:
\begin{enumerate}
  \item Abstract
  \item Introduction
  \item Literature review
  \item Method section: Initial algorithm and results
  \item Method section: Algorithm revision and results 
  \item Method section: Fully improved approach
  \item Discussion
  \item Conclusion
  \item Bibliography / References
\end{enumerate}
\subsection{Timetable}
The proposed timetable for successful completion of this project is as follows:

\begin{center}
  \begin{tabular}{||c | c||}
  \hline
  Date & Milestone \\ [0.5ex]
  \hline\hline
  16/04 & Explore branching \\
  \hline
  28/05 & Explore node selection \\
  \hline
  04/06 & Interim presentation, Lit. Rev  \\
  \hline
  30/07 & Explore warm starting \\
  \hline
  27/08 & Generalisations of method \\
  \hline
  10/09 & Problem type evaluation \\
  \hline
  01/10 & Thesis drafting process \\
  \hline
  29/10 & Final presentation \\
  \hline 
  05/11 & Thesis submission \\ [1ex]
  \hline
  \end{tabular}
\end{center}

Of course, as the project progresses, this timetable may be subject to change.

\subsection{Potential Difficulties}
As is the case with any research project, there are potential setbacks which should be acknowledged. Further, work on the project until this point has indicated that the following issues have the capacity to disrupt research:
\begin{itemize}
  \item Adaptations of known AAs to our scheme; while some methods are known, optimising and weaving individual AAs into our Branch-and-Bound may be both difficult, as being unable to integrate these well into could directly impact our results.
  \item The bridging of theoretically proven approximation algorithms to practical applications: algorithms specified for theoretical conclusions can predicate themselves on design choices unsuitable for practical application.
  \item Development of key project questions and ideas not addressed in adequate depth. This could be a result of both poor understanding and issues with planning.
  \item Devising strong theoretical basis for algorithmic design choices, when extending and improving algorithms.
  \item Difficulty in using secondary texts related to specific functionality, i.e. if there is coverage of the nuances of approximating a particular problem, it is possible that obtaining the source may be difficult due to limited copies.
\end{itemize}

\section{Deliverables / Outcomes}
We expect there to be a series of deliverables associated with the completion of this project. In addition to this proposal, a literature review, an interim and final presentation, and a final thesis will be provided. Moreover, as is expected with design science research approaches, we aim to have publicly available versions of our software for each point in the research, hosted online. We plan on making our implementation available as a public library upon completion of the project. 

\section{Bibliography / References}
\bibliography{proposal}
\bibliographystyle{plain}
\end{document}
